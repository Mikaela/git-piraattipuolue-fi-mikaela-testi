% This file is not directly from chatgpt as the package listings doesn't contain ini and chatgpt didn't provide proper configuration to make it recognise that, so I have stripped out `[language=ini]` on lines that begin `\begin{lstlisting}`

\documentclass{article}

\usepackage{amsmath} % Required for equation formatting
\usepackage{listings} % Required for code formatting

\title{Technical Documentation: System Configuration}
\author{Author Name}
\date{\today}

\begin{document}

\maketitle

\section{Introduction}

This technical documentation provides information on the configuration of a system. The system is comprised of multiple components, each of which must be properly configured in order to ensure optimal performance. This documentation will outline the steps necessary to properly configure each component of the system.

\section{System Components}

The system is comprised of the following components:

\begin{itemize}
    \item Component 1
    \item Component 2
    \item Component 3
\end{itemize}

\section{Component 1 Configuration}

Component 1 must be properly configured in order to ensure the system functions correctly. To configure Component 1, follow these steps:

\begin{enumerate}
    \item Step 1
    \item Step 2
    \item Step 3
\end{enumerate}

\section{Component 2 Configuration}

Component 2 must also be properly configured. To configure Component 2, follow these steps:

\begin{enumerate}
    \item Step 1
    \item Step 2
    \item Step 3
\end{enumerate}

\section{Component 3 Configuration}

The configuration of Component 3 is similar to that of Components 1 and 2. To configure Component 3, follow these steps:

\begin{enumerate}
    \item Step 1
    \item Step 2
    \item Step 3
\end{enumerate}

\section{Code Examples}

The following code examples provide further information on how to properly configure each component of the system.

\subsection{Component 1 Configuration Example}

\begin{lstlisting}
; Configuration file for Component 1

; Section 1
parameter1 = value1
parameter2 = value2

; Section 2
parameter3 = value3
parameter4 = value4
\end{lstlisting}

\subsection{Component 2 Configuration Example}

\begin{lstlisting}[language=html]
<!-- Configuration for Component 2 -->

<!-- Section 1 -->
<parameter1 value="value1"/>
<parameter2 value="value2"/>

<!-- Section 2 -->
<parameter3 value="value3"/>
<parameter4 value="value4"/>
\end{lstlisting}

\subsection{Component 3 Configuration Example}

\begin{lstlisting}
; Configuration file for Component 3

; Section 1
parameter1 = value1
parameter2 = value2

; Section 2
parameter3 = value3
parameter4 = value4
\end{lstlisting}

\section{Conclusion}

This technical documentation has outlined the steps necessary to properly configure each component of the system. By following the steps outlined in this documentation, the system should function correctly. If you have any questions or issues with the configuration of the system, please consult the manufacturer or seek professional assistance.

\end{document}
